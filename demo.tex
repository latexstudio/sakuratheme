\documentclass[11pt]{beamer}
\usepackage{lmodern}
\usepackage{amsmath,amssymb,bm,mathrsfs}
\usepackage{mathtools}
\usepackage{tikz}
\usetikzlibrary{automata,shapes}
\usepackage{calc}
\usepackage{cases}
\usepackage[many]{tcolorbox}
\usepackage{tabularx}
\usepackage{booktabs}
\newcolumntype{C}[1]{>{\hsize=#1\linewidth\centering\arraybackslash}X}
\newcolumntype{L}[1]{>{\hsize=#1\linewidth\raggedright\arraybackslash}X}
\newcolumntype{R}[1]{>{\hsize=#1\linewidth\raggedleft\arraybackslash}X}
\usetheme{sakura}
\usefonttheme{professionalfonts}
\makeatletter
\newif\if@japanese%
\@japanesetrue%
\if@japanese%
    \usepackage{luatexja-ruby}
\ltjsetparameter{jacharrange={-2,-3}}
\ltjsetparameter{alxspmode={`#,allow}}

\fi
\makeatother
\usepackage{exscale}
\usepackage[sort]{natbib}
\renewcommand{\bibsection}{}
\let\oldbibitem=\bibitem
\renewcommand{\bibitem}[2][]{\label{#2}\oldbibitem[#1]{#2}}
\let\oldcite=\citet
\renewcommand\citet[1]{\hyperlink{#1}{\oldcite{#1}}}
\let\oldcitep=\citep
\renewcommand\citep[1]{\hyperlink{#1}{\oldcitep{#1}}}
\usepackage{indent}
\usepackage{framed}
\usepackage{udline}
\usepackage{listings}
\usepackage[noend]{algpseudocode}
\everymath{\displaystyle}
\tcbset{%
    coltext=sDarkGray,
    colback=sLightGray,
}
\DeclareMathOperator*{\argmax}{arg\,max}
\DeclareMathOperator*{\argmin}{arg\,min}
\DeclareMathOperator*{\cov}{cov}
\DeclareMathOperator*{\sign}{sign}
\newcommand{\absolute}[1]{\left|#1\right|}
\newcommand{\parentheses}[1]{\left(#1\right)}
\newcommand{\braces}[1]{\left\{#1\right\}}
\newcommand{\brackets}[1]{\left[#1\right]}
\newcommand{\anglebrackets}[1]{\left\langle#1\right\rangle}
\newcommand{\norm}[1]{\left\|#1\right\|}
\newcommand{\mathif}[3]{\mathtt{if}\ #1\ \mathtt{then}\ #2\ \mathtt{else}\ #3}
\newcommand{\const}{\mathrm{const.}}
\newcommand{\condp}[2]{P\left(#1\,\middle|\,#2\right)}

\newcommand\enumref[1]{\textcolor{sRed}{\ref{#1}}}
\newcommand\thc[1]{\multicolumn{1}{c}{\textbf{#1}}}
\newcommand\marker[1]{\colorbox{sYellow}{#1}}
\newcommand\bluemarker[1]{\colorbox{sLightBlue}{#1}}
\newenvironment{xleftbar}[1][\hsize]{%
    \def\FrameCommand{\vrule width 3pt \hspace{10pt}}%
    \MakeFramed{\hsize#1\advance\hsize-\width\FrameRestore}%
}
{\endMakeFramed}
\tikzset{%
    state/.style={%
        draw,
        minimum height=2em,
        inner sep=5pt,
        text centered,
    }
}
\hypersetup{%
    unicode=true,
    backref=true,
    hidelinks=true
}
\pgfkeys{/metropolis/outer/.cd,
    numbering=fraction,
    progressbar=none
}
\makeatletter
\newcommand*{\getlength}[1]{\strip@pt#1}
\makeatother
\usepackage{toneletter}
\usepackage{gb4e,cgloss4e}
\noautomath%
\title{主題}
\subtitle{副題}
\institute{所属}
\author{名前}
\makeatletter
\if@japanese%
    \date{{\number\year}年{\number\month}月{\number\day}日}
\else
    \date{\today}
\fi
\makeatother
\begin{document}

\begin{frame}
    \nocite{demo}
    \maketitle
\end{frame}

\section{言語学関連}
\subsection{例文と対訳}
\begin{frame}
\frametitle{番号付きの例文}
    \begin{exe}
        \ex%
        \glll {辞書} {を} {食べた。} \\
              {jisho} {wo} {tabeta} \\
              {dictionary} {\textsc{acc}} {eat+\textsc{past}} \\
              \trans {“(I) ate an dictionary.”}
    \end{exe}
\end{frame}

\subsection{音声}
\begin{frame}
    \frametitle{音声}
    Unicode入力で,IMEに音声記号を入力するための手段があれば,
    \texttt{tipa}パッケージは必要ありません.

    ただし,基準線を用いた声調記号については\texttt{tipa}を用いたほうが便利だと思います.
    声調記号を扱うスタイルファイル\texttt{tone.sty}の一部を切り出して変更したものを
    \texttt{toneletter.sty}としてレポジトリに置いてあるので,\texttt{toneltter}などのように使うと
    他のパッケージとの衝突を避けられると思います.
    \begin{table}
        \centering
        \caption{標準中国語の声調}
        \begin{tabular}{cccc}
            T1 & T2 & T3 & T3 \\
            ma\toneletter{55} &
            ma\toneletter{35} &
            ma\toneletter{214} &
            ma\toneletter{51} \\
        \end{tabular}
    \end{table}
\end{frame}

\section{数学関連}
\begin{frame}
\frametitle{積分}
\begin{align}
    \int_{-\infty}^\infty \frac{1}{\sqrt{2\pi}\sigma}\exp\parentheses{-\frac{\parentheses{x - \mu}^2}{2\sigma^2}} dx = 1
\end{align}
\end{frame}

\begin{frame}
\frametitle{日本語で書かれた文書の引用}
\texttt{{\textbackslash}cite\{key\}}で\citet{demo}や\citet{japanese}のように引用されます.
    \begin{itemize}
        \item PDF上で
        \item 参考文献を
        \item クリックすると
        \item 参考文献一覧に飛びます
    \end{itemize}
\end{frame}

\section{まとめ}
\begin{frame}
\frametitle{まとめ}
まとめ
\end{frame}

\begin{frame}[allowframebreaks]
\frametitle{参考文献}
\begingroup
\scriptsize
    \setbeamertemplate{bibliography item}[triangle]
    \bibliographystyle{j}
    \bibliography{demo}
\endgroup
\end{frame}

\end{document}
