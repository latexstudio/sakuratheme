\documentclass[11pt]{beamer}
\usepackage{lmodern}
\usepackage{amsmath}
\usepackage{tikz}
\usepackage{booktabs}
\usetheme{sakura}
\usefonttheme{professionalfonts}
\makeatletter
\newif\if@japanese%
\@japanesetrue%
\makeatother
\usepackage{luatexja-ruby}
\ltjsetparameter{jacharrange={-2,-3}}
\ltjsetparameter{alxspmode={`#,allow}}
 % charrange などの設定
\makeatother
\usepackage{exscale}
\usepackage[sort]{natbib}
\renewcommand{\bibsection}{}
\let\oldbibitem=\bibitem
\renewcommand{\bibitem}[2][]{\label{#2}\oldbibitem[#1]{#2}}
\let\oldcite=\citet
\renewcommand\citet[1]{\hyperlink{#1}{\oldcite{#1}}}
\let\oldcitep=\citep
\renewcommand\citep[1]{\hyperlink{#1}{\oldcitep{#1}}}
\usepackage{indent}
\usepackage{framed}
\usepackage{listings}
\usepackage[noend]{algpseudocode}
\everymath{\displaystyle}
\DeclareMathOperator*{\argmax}{arg\,max}
\DeclareMathOperator*{\argmin}{arg\,min}
\DeclareMathOperator*{\cov}{cov}
\DeclareMathOperator*{\sign}{sign}
\newcommand{\absolute}[1]{\left|#1\right|}
\newcommand{\parentheses}[1]{\left(#1\right)}
\newcommand{\braces}[1]{\left\{#1\right\}}
\newcommand{\brackets}[1]{\left[#1\right]}
\newcommand{\anglebrackets}[1]{\left\langle#1\right\rangle}
\newcommand{\norm}[1]{\left\|#1\right\|}
\newcommand{\mathif}[3]{\mathtt{if}\ #1\ \mathtt{then}\ #2\ \mathtt{else}\ #3}
\newcommand{\const}{\mathrm{const.}}
\newcommand{\condp}[2]{P\left(#1\,\middle|\,#2\right)}

\usepackage{arabluatex}
\newfontfamily\arabicfont[Script=Arabic]{Noto Naskh Arabic}
\newcommand\rtltriangle{\scriptsize\raise1.25pt\hbox{\donotcoloroutermaths$\blacktriangleleft$}}
\newcommand\header[1]{\multicolumn{1}{c}{\textbf{#1}}}
\newcommand\marker[1]{\colorbox{sYellow}{#1}}
\newcommand\bluemarker[1]{\colorbox{sLightBlue}{#1}}
\tikzset{%
    state/.style={%
        draw,
        minimum height=2em,
        inner sep=5pt,
        text centered,
    }
}
\hypersetup{%
    unicode=true,
    backref=true,
    hidelinks=true
}
\pgfkeys{/metropolis/outer/.cd,
    numbering=fraction,
    progressbar=none
}
\makeatletter
\newcommand*{\getlength}[1]{\strip@pt#1}
\makeatother
\usepackage{toneletter}
\usepackage{gb4e,cgloss4e}
\noautomath%
\title{主題}
\subtitle{副題}
\institute{所属}
\author{名前}
\makeatletter
\if@japanese%
    \date{{\number\year}年{\number\month}月{\number\day}日}
\else
    \date{\today}
\fi
\makeatother
\begin{document}

\begin{frame}
    \nocite{demo}
    \maketitle
\end{frame}

\section{言語学関連}
\subsection{例文と対訳}
\begin{frame}
\frametitle{番号付きの例文}
    \begin{exe}
        \ex%
        \glll {辞書} {を} {食べた。} \\
              {jisho} {wo} {tabeta} \\
              {dictionary} {\textsc{acc}} {eat+\textsc{past}} \\
              \trans {“(I) ate an dictionary.”}
    \end{exe}
\end{frame}

\subsection{音声}
\begin{frame}
    \frametitle{音声}
    Unicode入力で,IMEに音声記号を入力するための手段があれば,
    \texttt{tipa}パッケージは必要ありません.

    ただし,基準線を用いた声調記号については\texttt{tipa}を用いたほうが便利だと思います.
    声調記号を扱うスタイルファイル\texttt{tone.sty}の一部を切り出して変更したものを
    \texttt{toneletter.sty}として使うと他のパッケージとの衝突を避けて使用できます.

    \bigskip

    \begin{table}
        \centering\small
        \caption{標準中国語の声調}
        \begin{tabular}{cccc}
            % \toprule, \midrule, and \bottomrule need booktabs
            \toprule
            \header{T1} & \header{T2} & \header{T3} & \header{T3} \\
            \midrule
            ma\toneletter{55} &
            ma\toneletter{35} &
            ma\toneletter{214} &
            ma\toneletter{51} \\
            \bottomrule
        \end{tabular}
    \end{table}
\end{frame}

\begin{frame}
    \frametitle{長め文章の引用}
    \begin{leftbar}
        「あの森\ruby{琴}{ライラ}の宿でせう。あたしきつとあの森の中には、むかしの大きなオーケストラの人たちが
        集まつていらつしやると思ふわ。まはりには青い孔雀やなんかたくさんゐると思ふわ。」
        女の子が答へました。\hfill 宮澤賢治「銀河鉄道の夜」
    \end{leftbar}
\end{frame}

\begin{frame}

    \frametitle{ラテン文字、キリル文字、アラビア文字}
    \setlength{\leftmargini}{1em}
    \begin{itemize}
        \item Falsches Üben von Xylophonmusik quält jeden größeren Zwerg

        \item Широкая электрификация южных губерний даст мощный толчок подъёму сельского хозяйства.
        % Temporary solution
        \setRL
        \item[\rtltriangle]\hspace{1.2em}
            \arb[fullvoc]{.sif
                xalqa xawdiN kami_tli 'l-^samsi 'i_d bana .gat
                \LR{\bfseries ---} ya.h.zY_a 'l-.dajI`u bihA najlA|"'a mi`.tAri}
    \end{itemize}
\end{frame}

\section{数学関連}
\begin{frame}
\frametitle{積分}
\begin{align}
    \int_{-\infty}^\infty \frac{1}{\sqrt{2\pi}\sigma}\exp\parentheses{-\frac{\parentheses{x - \mu}^2}{2\sigma^2}} dx = 1
\end{align}
\end{frame}

\begin{frame}
\frametitle{日本語で書かれた参考文献の引用}
\texttt{{\textbackslash}cite\{key\}}で\citet{demo}や\citet{japanese}のように引用されます.
    \begin{itemize}
        \item PDF上で
        \item 参考文献を
        \item クリックすると
        \item 参考文献一覧に飛びます
    \end{itemize}
\end{frame}

\section{まとめ}
\begin{frame}
\frametitle{まとめ}
まとめ
\end{frame}

\begin{frame}[allowframebreaks]
\frametitle{参考文献}
\begingroup
\scriptsize
    \setbeamertemplate{bibliography item}[triangle]
    \bibliographystyle{j}
    \bibliography{demo}
\endgroup
\end{frame}

\end{document}
